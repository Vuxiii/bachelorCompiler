Let us dive into the backend of the compiler. Here we define the semantics of our language. At this stage we have the AST from the parser which we will traverse multiple times in order to collect various components and information about the source code, which determine what should happen at runtime. This stage can be divided into multiple phases just like the frontend. For this compiler the phases are as follows: 
\begin{enumerate}
    \item Symbol Collection
    \item String Collection
    \item Register The Stack's Layout
    \item Register The Structs' Layout
    \item Intermediate Code Generation
    \item Code Emit
\end{enumerate}


use to generate IR (intermediate representation). IR is a construct designed to make it easier to manipulate and handle the code during construction. For this project the IR is a set of instructions and 

\section{Symbol Collection}
The goal of the symbol collector is to collect the symbols defined in the source code and map them accordingly. Symbols are variables, struct definitions and function names. Variables are usually placed on the stack unless it has been heap allocated. We traverse the AST and collect all declarations one by one. 

This phase is also where the compiler checks for scoping rules. I have chosen to implement static scoping, and this means that during an inorder traversal of the AST if at any point a variable is referenced, which have yet to be declared, we have encountered a scoping error. The symbol has been referenced before being declared. Imagine the following scoping example:

\begin{lstlisting}[language=Python,title=Static Scoping]
let fun1: ();
let var1: int;
var1 = 42;

fun1 = () {
    let var2: int;
    var2 = var1 + 4;
    print("Variable 2 is: %\n", var2);
};
\end{lstlisting}

In the above example at line 7 there currently are two stack frames. One for the outer scope, and one for the scope of the function \textit{fun1}. In order to read the value of \textit{var1} we first need to go to the outer frame, and then find the correct offset for that specific variable on the stack. For that reason we construct a lookup table to see if we are in the same scope as the referenced symbol. If we are not, we need to locate it. This can be done by following the old basepointer stored on the stack and recursivly check if we are in the correct scope.  

\section{String Collection}

This collector traverses the AST to locate any strings that are used in the source code. A program can quickly use alot of strings, and some of which are repeated multiple times. For space efficiency reasons we want to only store unique strings in the final output code.

\subsection{Garbage Collector}
Arguably a garbage collector is not a part of the compiler, however, the metadata that the garbage collector needs will be collected in the compiler.


For the garbage collector to function we need to register all the pointers located on the stack. Additionally we need to go through all user defined structs and identify which have pointers and what offset these pointers have from the start of the memory block. 

Let us call this for a \textit{Layout}. A layout is a series of bitfields indicating with a one if that specific offset is a pointer, or a zero if it can be ignored. For this we can use 64 bit bitfields, where the first bit indicates if there is another bitfield after this. This is necessary for large structs or stack frames, where there are more than 63 pointers or variables. Imagine the following struct definition:

\begin{lstlisting}[language=C,keywords={Thing, int},title=Type Definition]
type Thing: {
    let a: *int;
    let b: int;
    let c: int;
    let d: *int;
};
\end{lstlisting}

In the above example the bitfield for the struct Thing will have the first bit set to zero, because there are less than 64 fields the struct. The second and the fifth bits are set to one and the rest zero. 

\subsection{Stack Layout}
We will now discuss how to collect the pointers located on the stack. For each possible stack frame we have to generate a layout which keeps track of where a pointer is located in respect to the currently active stack frame. Stack frames occur in functions and in the main scope. This means that we traverse the main scope to identify its layout, and for each function do the same. For each traversal of the AST we locate any pointer declarations to the heap. Since the symbol collector already is mapping offsets for each local variable, we can use this offset counter to mark pointers on the stack. 

\subsection{Struct Layout}

Just like generating layouts for each unique stack frame we generate layouts for each unique struct. Whenever we find a struct declaration we register any pointers it may contain, and depending on whether this specific struct has been allocated on the stack or on the heap, the pointer offsets will be negative or positive respectivly. 

\section{Intermediate Representation}

Now that the components needed to construct the emitted code have been collected, we can start to define the semantics of the language. At this point we will convert the AST to IR which is an internal representation of the output code. For this compiler the IR is very close to X86--64 assembly. The IR is divided into an opcode and a list of operands depending on the opcode. Additionally it can contain comments to make it easier to understand what the compiler has emitted. The advantage of going through this step is that later on different optimization strategies can be applied such as peephole optimization, which identifies patterns of opcodes and replaces them with more efficient opcodes. An example of this could be a sequence of a push followed by a pop. This sequence can be replaces by a simple move instruction instead.  

\section{Constructing the Language Semantics}

At this point we have some options. We can go through the AST and start by constructing the IR for each function and then assemble the code located in the main scope. This will result in multiple lists of IR code. By choosing this path it will be easier in the code emit phase. Another option is to go through the AST line by line and construct everything inorder. This brings some disadvantages. We have no seperation of the main code and functions. Another huge consideration is parallalization of the IR code construction. With the latter choice it would be very difficult to find different compilation units that can be compiled seperatly. Using the former technique we can divide each function assembly as its own compilation unit and harness the power of the computer's cores. Therefore let us choose that option. Because each function definition is a subtree it is easy to isolate them. 

Let us now go through how we can construct different semantics that high level languages contain. 

\begin{itemize}
    \item \textbf{Variables}: Variables needs two actions that can be performed with them. Firstly it needs to be able to store some data. Secondly we need to be able to read the value it stores. The symbol collector has already chosen a spot on the stack for each variable. 
    \item \textbf{Structs}: Structs are a collection of variables located right next to each other. For that reason fields in a struct can simply be though of as variables with a longer name.
    \item \textbf{Functions}: Functions are seperate blocks of code that can be called throughout the execution of the program. Each function will have a label that the program can jump to when a function call to that function has been reached. For the functions footer it needs a return such that the assembler can clean up after itself and return to the calling code.
    \item \textbf{Branching}: Branching is a sequence of conditions that when met execute a block of code. This language supports if statements, if else statements, and if else if statements. If the first condition fails, the runtime checks the next condition until it has exhausted the list of if statements. If the list contains an else statement that block of code will be executed if none of the prior conditions were met. After a successful condition and the execution of the block has finished, it jump to the end of the list statements. This ensures that only a single block of code will be executed.
    \item \textbf{Printing}: This language supports string interpolation in print statements. This works by printing the string in slices and inserting the specified input in between the slices. The arguments to be interpolated are evaluated from left to right and inserted in that order into the string.
\end{itemize}

\section{Code Emit}

The variables will then be placed in order on the stack after the base pointer. Functions will be placed outside the main entry point in the assembly, such that these are callable later on. Struct definitions are essentially just a bunch of variables packed together and placed contiguously on the stack. 