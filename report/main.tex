\documentclass[a4paper,12pt]{report}

% Packages
\usepackage[utf8]{inputenc}
\usepackage{titling}
\usepackage{fancyhdr}
\usepackage{graphicx}
\usepackage{parskip}
\usepackage{amsmath}
\usepackage{amssymb}
\usepackage{xspace}

\usepackage{tikz}
\usetikzlibrary{automata,positioning}
\usepackage{subfiles}
\usepackage{listings}
\usepackage{xcolor}
\usepackage{subfig}

\definecolor{codegreen}{rgb}{0,0.6,0}
\definecolor{codegray}{rgb}{0.5,0.5,0.5}
\definecolor{codepurple}{rgb}{0.58,0,0.82}
\definecolor{backcolour}{rgb}{0.95,0.95,0.92}
\lstdefinestyle{mystyle}{
    backgroundcolor=\color{backcolour},   
    commentstyle=\color{codegreen},
    keywordstyle=\color{magenta},
    numberstyle=\tiny\color{codegray},
    stringstyle=\color{codepurple},
    basicstyle=\ttfamily\footnotesize,
    breakatwhitespace=false,         
    breaklines=true,                 
    captionpos=b,                    
    keepspaces=true,                 
    numbers=left,                    
    numbersep=5pt,                  
    showspaces=false,                
    showstringspaces=false,
    showtabs=false,                  
    tabsize=2    
}

\lstset{style=mystyle}


% Algorithms
%\usepackage{algorithm}
%\usepackage[noend]{algorithmic}
%\usepackage{listings}

% Tabular
%\usepackage{multirow}
%\usepackage{rotating}

% Figures
%\usepackage{epic,eepic}

% Consider periods after theorems and "Proof" and consider not forcing Roman.
% Theorems
\newtheorem{xdefinition}{Definition}
\newtheorem{xobservation}{Observation}
\newtheorem{xtheorem}{Theorem}
\newtheorem{xlemma}{Lemma}
\newtheorem{xproposition}{Proposition}
\newtheorem{xcorollary}{Corollary}
\newenvironment{definition}{\begin{xdefinition}\rm}%
{\hspace*{\fill}\raisebox{-1pt}{\boldmath$\Box$}\end{xdefinition}}
\newenvironment{observation}{\begin{xobservation}\rm}%
{\hspace*{\fill}\raisebox{-1pt}{\boldmath$\Box$}\end{xobservation}}
\newenvironment{theorem}{\begin{xtheorem}\rm}{\end{xtheorem}}
\newenvironment{lemma}{\begin{xlemma}\rm}{\end{xlemma}}
\newenvironment{proposition}{\begin{xproposition}\rm}{\end{xproposition}}
\newenvironment{corollary}{\begin{xcorollary}\rm}{\end{xcorollary}}
\newenvironment{proof}{\begin{trivlist}\item[]{\bf Proof }}%
{\hspace*{\fill}\raisebox{-1pt}{\boldmath$\Box$}\end{trivlist}}

% Example commands
\newcommand{\OPT}{\ensuremath{\operatorname{\textsc{Opt}}}\xspace}
\newcommand{\MIN}[1]{\min\left\{#1\right\}}
\newcommand{\CEIL}[1]{\left\lceil#1\right\rceil}
\newcommand{\FLOOR}[1]{\left\lfloor#1\right\rfloor}
\newcommand{\SIZE}[1]{|#1|}
\newcommand{\SET}[1]{\left\{#1\right\}}
\newcommand{\SETOF}[2]{\SET{#1 \mid #2}}
\newcommand{\BIGSET}[1]{\left\{#1\right\}}
\newcommand{\BIGSETOF}[2]{\BIGSET{#1 \mid #2}}
\newcommand{\NAT}{\ensuremath{\mathbb{N}}}
\newcommand{\HRule}{\rule{\linewidth}{0.5mm}}

\newcommand{\term}[1]{\text{t\textunderscore #1}}
\newcommand{\nont}[1]{\text{n\textunderscore #1}}

%%%%%%%%%%%%%%%%%%%%%%%%%%%%%%%%%%%%%%%%%%%%%%%%%%%%%%%%%%%%%%%%%%%%%%%%%%%

\begin{document}
		
\begin{titlepage}

\center % Center everything on the page

\textsc{\LARGE Department of Mathematics}\\[0.3cm]

\textsc{\LARGE and  Computer Science}\\[1.5cm]

\textsc{\Large University of Southern Denmark}\\[0.5cm]

\textsc{\large Bachelor Thesis in Computer Science }\\[0.5cm]
% \textsc{\large Master Thesis in Computer Science }\\[0.5cm]

\HRule\\[0.4cm]

{\huge\bfseries A Compiler for the Juhl language}\\[0.4cm]

\HRule\\[1.5cm]

\begin{minipage}{0.47\textwidth}
        \begin{flushleft}
                \large
                \textit{Author}\\
                William Juhl \\ 
                wijuh20@student.sdu.dk
        \end{flushleft}
\end{minipage}
%~ % Uncomment from this line if there are two students
%\begin{minipage}{0.47\textwidth}
%        \begin{flushleft}
%                \large
%                \textit{Author}\\
%                Your name \\ 
%                youremail@student.sdu.dk
%        \end{flushleft}
%\end{minipage}

\vfill

\begin{minipage}{0.45\textwidth}
        \begin{flushleft}
                \large
                \textit{Supervisor}\\
                Kim Skak Larsen\\
                Professor
        \end{flushleft}
\end{minipage}
%~ % Uncomment from this line if there are two supervisors
%\begin{minipage}{0.45\textwidth}
%        \begin{flushleft}
%                \large
%                \textit{Supervisor}\\
%                Excellent Additional Help\\
%                Title
%        \end{flushleft}
%\end{minipage}

\vfill\vfill

{\large\today} % At the very end, insert date for turning in, but leave it like this during the advising

\vfill\vfill
\includegraphics[width=0.4\textwidth]{SDU_BLACK_RGB.png}\\[1cm]

\vfill

\end{titlepage}

\pagenumbering{roman} 

\begin{abstract}
\paragraph{English}
your abstract in English

\paragraph{Danish}
your abstract in Danish
\end{abstract}

\newpage
\section*{Acknowledgments}
Thank you, Mom \& Dad!

\newpage
\tableofcontents

\newpage
\subfile{sections/abstract.tex}
\newpage
\pagenumbering{arabic} 
\setcounter{page}{1}

\chapter{Introduction}
\subfile{sections/introduction.tex}
\chapter{Tokenization}
\subfile{sections/tokenization.tex}
\chapter{Parsing}
\subfile{sections/parsing.tex}

\chapter{Compilation}
\subfile{sections/compilation.tex}
\chapter{Runtime Machine}
\subfile{sections/runtime.tex}
\chapter{Concluding Remarks}
\section{Further Improvements}


\bibliography{refs}
\bibliographystyle{plain}

\appendix

\newpage

\chapter*{Appendix: Lists of Stuff}
\label{appendix}
This is under development!

These issues are to some extent my personal opinions!

The topics are to a large extent guided by the type of mistakes Danish
students would make.

There are of course numerous issues that I do not discuss, some of
which are hard for Danes, such a subjunctive forms, rules (other
than the obvious) for the use of singular vs.\ plural forms, etc.

\section*{Matters of Style}

\subsection*{Miscellaneous}
\begin{itemize}
\item
Id est: use punctuation as
``\ldots,\textvisiblespace{}i.e.,\textvisiblespace''.
The construction should be followed by more than just one word,
usually a whole or partial sentence.
The construction should never start a sentence.
To be very strict, it should also not start the inside
of a parenthesis (use ``that is'' followed by a comma instead).
The same rules apply to ``e.g.''
If you insist on \emph{not} using a comma, make sure to use a backslash
to prevent \LaTeX{} from making an end-of-sentence space.
\item
Do not concatenate words or hyphenate them just because you would in Danish.
It is called ``type checker'', ``data structure'', ``garbage collector'', etc.
\item
Do not start a sentence with ``Else''; use ``Otherwise''.
\item
Do not start a sentence with ``But''; use ``However'', ``On the contrary'',
or something else appropriate.
\item
When capitalizing a headline, capitalize both words in a hyphenated
construction.
\item
Capitalize named entities such as ``Figure~42''.
\item
Unless one is comparing numbers, small numbers in text should be
spelled out.
\item
  There are (almost) always better (more precise) words than
  big, do, put, and things.
\item
In formal writing, do not use contractions such as ``don't''.
\item
  In formal writing, ``like'', in comparisons, should most often be ``such as''.
\item
  Do not use ``OK''.
\item
``Look at'' is informal and should often be replaced by ``Consider''.
\item
  Use ``essentially'' very sparingly. Never use it if something is
  actually exactly as described. Otherwise, it is often possible
  to explain or hint at the difference with few words instead of
  using ``essentially''.
\item
  I see many word order errors such as
  ``the decisions resulted always in more work'' instead of
  ``the decisions always resulted in more work''.
\item
  ``It'' cannot be the subject of something such as ``it is forbidden to smoke
  here''. 
\item
  Do not use ``less'' for countable items, i.e., ``Adam had less apples than Eve'' is incorrect and should be ``Adam had fewer apples than Eve''.
\end{itemize}

\subsection*{Commas}
\begin{itemize}
\item
Place a comma before ``then'' or instead of a ``then'' that has been left out
in an ``if-then'' construction. Also after the first part of ``since'' constructions and similar (where ``then'' is of course out of the question).
\item
  Use commas around ``however'' and similar constructions.
\item
  In lists, use comma after the penultimate item in the list, e.g., before ``and''. Also use a comma before ``etc''.
\item
  There are exceptions, but normally there should not be a comma before ``that''.
\item
  Use a comma after an introductory clause or word, just before a main clause.
  Main clauses joint by most conjunctions (``and'', ``but'', etc.)
  should also be separated by commas.
\end{itemize}

\section*{\LaTeX{}}
\begin{itemize}
\item
Use commands for most constructions and variable names other than
one letter math variables; see examples in this file.
\item
Remember backslash after abbreviation periods as
``vs.$\backslash$\textvisiblespace''.
\item
Never typeset names in math mode, i.e.,
``{\$}left{\$}'' is strictly forbidden. Use
``{\$}$\backslash{}$mathit\{left\}{\$}'', or just
``$\backslash{}$textit\{left\}'' if you do not need math mode.
This is how ugly is looks if you use math mode: $left$.
And this is when it is done correctly: \textit{left}.
\item
Use $\sim$ (space forbidding line break) between names of numbered
entities, e.g., write ``Figure$\sim$42'' to avoid that ``42'' can
appear at the start of a line. (Of course, you would never write ``42'',
but instead refer to a label.)
\item
Use one hyphen to hyphenate words and two hyphens to create the little line
between two numbers of an interval. A dash should be either two hyphens
surrounded by space or three hyphens and no space. It is a style issue
and a paper should only contain one type.
\item
Sometimes a style is enforced on you, but if you can decide
yourself, I find it much easier to read a text where paragraphs are
separated by vertical space than when a new paragraph is just indicated
by indentation. It is of course a style matter that should be defined
in the preamble.
\item
Use \verb"\cdots" for missing parts of a sequence when there are no commas;
if there are commas, as in $1, 2, \ldots, n$, use \verb"\ldots" to show
the continuation.
\end{itemize}

\section*{Writing Advice for TCS}
\begin{itemize}
\item
Think carefully about the naming of objects. When more than one object
of the same type is used, it is often a good idea to use letters close
to each other in the alphabet. That establishes a sort of typed language
that helps the reader.
However, skip letters with very established meaning, e.g., it is
often not good to use $f$ for something that is not a function.
Primes can also be used to stay in the same name space.
Also using first letters of the names of entities, equivalent Greek
letters, etc.\ can be helpful.
And do not give a name to a concept or introduce notation for it
unless you need it later.
\item
Sentences explaining progress in proof can be helpful: Where are we, and what
are we doing now?
\item
Do not say ``It is easy to see that \ldots''. If it is true, you
can almost always in the same space say ``Since this and that, \ldots''.
\item
Insert data unreduced into formulas and then use arithmetic,
so it is easy to see where things come from, i.e., do not combine
insertion and reduction steps.
\item
  Be careful with pronouns and be sure that it is clear what they refer to.
  Very often it is better to write the term ``it'' is referring to instead of writing ``it''.
\item
  Things that are the same should be written the same, so that one can more clearly see differences. Technical writing is different from writing literature! Also use a fixed word for any action, i.e., do not use put, place, write, etc. for
  the action.
\end{itemize}

\section*{Slides for the Presentation}
\begin{itemize}
\item
  Use the slides primarily for figures, graphs, and keywords. It is seldom a
  good idea to include many full sentences
  (except maybe a theorem statement) and
  almost never a good idea with full paragraphs of text.
\item
  Think about how you \emph{supplement} your written work,
  rather than just showing excerpts.
  For instance, if you show the workings of an algorithm, you can have the
  algorithm on one side and the result it is working on on the other side,
  and you can highlight lines of the algorithm as the results change
  incrementally.
\end{itemize}

\end{document}
